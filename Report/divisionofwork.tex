%!TEX root = report.tex

\clearpage
\section*{Appendix}
\subsection*{Division of Work}\label{sec:division}

In the first three months, when we built the simulation with the five of us, the work was actually pretty well divided and everybody did their part. At this time, Travis worked on the code for fire propagation while Dirk Jelle hunted bugs and tested the simulator. When the group split up and Travis and Dirk Jelle decided to work together, Travis had already ported the simulation from Java to Python. In the next few weeks, Travis did the main part of the work of exploring the OpenAI Gym environments and implementing a basic DQN algorithm. Dirk Jelle kept up with Travis' progression, but was focussed on his last exams at that time.

When the algorithm was able to learn, both of us explored what it could and could not do by running tests and tweaking parameters. We each implemented our own modification solo: Travis did SARSA and Dirk Jelle did Dueling Networks. Once we had an idea of what tests we wanted and needed to run, Dirk Jelle spent the most time learning how to use Peregrine. The work regarding the generating and processing of the results was mainly done by Dirk Jelle.

The work required for the presentation at the Bachelor's Symposium was divided equally. However, since Travis focussed more on the theory and the algorithm, he presented that part. Also Dirk Jelle presented what he was most familiar with, which were the simulation and the results.

An important note is that the neither of us excluded the other when doing some part of the project. We kept in touch at least every few days and always had the opportunity to ask questions to each other. While Travis wrote the bulk of the code for the learning algorithm, we both discussed how it should work and why. When Dirk Jelle processed the data to create results, we both spoke about what data goes into which type of plots/tables and why.

In conclusion, this project proved to be a lot of work. While we tried our best to do every task together, we did not always succeed. In the end, neither of us feel like the total workload was divided unfairly. Both of us worked hard, but not always at the same time. We hope this small summary is able to give some insight into who did what for the last six months.
